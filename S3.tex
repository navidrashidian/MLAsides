% arguelles v2.3.0
% author: Michele Piazzai
% contact: michele.piazzai@uc3m.es
% license: MIT

% Copied from GitHub: https://github.com/piazzai/arguelles

\documentclass[compress,12pt]{beamer}

\usetheme{Arguelles}

\title{Frege's Logicism}
\subtitle{Proof: A thematic history \#3}
\event{}
\date{}
\author{Navid Rashidian}
\institute{University of Tehran}
%\email{}
%\homepage{}
%\github{}

\newcommand*{\is}{\ \ -\mkern-7mu<}
\newcommand*{\add}{+\mkern-4mu,}

\begin{document}

\frame[plain]{\titlepage}

\Section{Begriffsschrift}

\begin{frame}
    \frametitle{Two Kinds of Truth}
    \begin{quote}
        [W]e divide all truths that require justification into two kinds, those whose proof can be given purely logically and those whose proof must be grounded one mpirical facts. But there is no inconsistency in a proposition belonging to the first kind and yet being such that it can never be apprehendedby a human mind without the operation of the senses. (BS, III)
    \end{quote}
\end{frame}

\begin{frame}{title}
    \frametitle{The need for a \emph{Begriffsschrift}}
    \begin{quote}
        Now in considering the question of to which of these two kinds arithmetical judgements belong, I first had to see how far one could get in arithmetic by inferences alone, supported only by the laws of thought that transcend all particulars. The course I took was first to seek to reduce the concept of ordering in a series to that of logical consequence, in order then to progress to the concept of number. So that nothing intuitive could intrude here unnoticed, everything had to depend on the chain of inference being free of gaps. In striving to fulfil this requirement in the strictest way, I found an obstacle in the inadequacy of language: however cumbersome the expressions that arose, the more complicated the relations became, the less the precision was attained that my purpose demanded. Out of this need came the idea of the present \textup{Begriffsschrift}. (BS, IV)
    \end{quote}
\end{frame}

\begin{frame}{title}
    \frametitle{Judgement}
    \begin{quote}
        
    \end{quote}
\end{frame}

\begin{frame}{title}
    \frametitle{Conditionality}
    \begin{quote}
        
    \end{quote}
\end{frame}

\begin{frame}{title}
    \frametitle{Negation}
    \begin{quote}
        
    \end{quote}
\end{frame}

\begin{frame}{title}
    \frametitle{Predicates and Relations}
    \begin{quote}
        
    \end{quote}
\end{frame}

\begin{frame}{title}
    \frametitle{Quantifiers}
    \begin{quote}
        
    \end{quote}
\end{frame}

\Section{Grundlagen}

\begin{frame}
    \frametitle{Euclidean rigour}
    \begin{quote}
        After departing for a long time from Euclidean rigour,mathematics is now returning to it, and even striving to take it further. Inarithmetic, simply as a result of the origin in India of many of its methodsand concepts, reasoning has traditionally been less strict than ingeometry, which had mainly been developed by the Greeks... Later developments, however, haveshown more and more clearly that in mathematics a mere moralconviction, based on many successful applications, is insufficient. A proof isnow demanded of many things that previously counted as self-evident...  This path must eventually lead to the concept of Number and the simplest propositions holding of the positive whole numbers, which form the foundation of the whole of arithmetic. (Gl, \S1--2)
    \end{quote}
\end{frame}

\begin{frame}
    \frametitle{Numbers as Second-Order Concepts}
    \begin{quote}
        This suggests as the answer to the first question of the previous section that a statement of number contains an assertion about a concept. This is perhaps clearest in the case of the number 0. If I say 'Venus has 0 moons', then there is no moon or aggregate of moons to assert anything of at all; but instead it is the concept 'moon of Venus' to which a property is ascribed, namely, that of including nothing under it. If I say 'The King's carriage is drawn by four horses', then I am ascribing the number four to the concept 'horse that draws the King's carriage'. (Gl, \S46)
    \end{quote}
\end{frame}

\begin{frame}
    \frametitle{Defining $0$, $1$ and $n+1$}
    \small
    \begin{quote}
        It is natural to say: the number $0$ belongs to a concept if no object falls under it. But this appears to replace $0$ by 'no', which means the same. The following formulation is therefore preferable: the number $0$ belongs to a concept if, whatever $a$ may be, the proposition holds universally that $a$ does not fall under that concept. In a similar way we could say: the number $1$ belongs to a concept $F$ if, whatever a may be, the proposition does not hold universally that adoes not fall under $F$, and if from the propositions
        
        '$a$ falls under $F$' and '$b$ falls under $F$'
        
        it follows universally that a and b are the same.
        
        It still remains to give a general definition of the transition from one number to the next. We could try the following formulation: the number $(n + 1)$ belongs to the concept $F$ if there is an object $a$ falling under $F$ such mat the number $n$ belongs to the concept 'falling under $F$, but not $a$'.20 (Gl, \S55)
    \end{quote}
\end{frame}

\begin{frame}
    \frametitle{The Julius Caesar Problem}
    \begin{quote}
        We can, of course, by means of this and the second definition say what is meant by
        
        'the number $1 + 1$ belongs to the concept $F$',
        
        and then, using this, give the sense of the expression
        
        'the number $1 + 1 + 1$ belongs to the concept $F$
        
        and so on; but we can never---to take an extreme example---decide by means of our definitions whether the number \textup{Julius Caesar} belongs to a concept, or whether that well-known conqueror of Gaul is a number or not. Furthermore, we cannot prove with the help of our attempted definitions that if the number $a$ belongs to the concept $F$ and the number $b$ belongs to the same concept, then necessarily $a = b$.  (Gl, \S56)
    \end{quote}
\end{frame}

\begin{frame}
    \frametitle{Equinumerosity}
    \begin{quote}
        If the symbol $a$ is to designate an object for us, then we must have a criterion thatdecides in all cases whether $b$ is the same as $a$, even if it is not always in our power to apply this criterion. In our case we must define the sense ef the proposition
        
        'The number that belongs to the concept $F$ is the same as the number that belongs to the concept $G$'[.]

        Hume has already mentioned such a means: 'When two numbers are so combined, as that the one has always a unit answering to every unit of the other, we pronounce them equal'. (Gl, \S62--3)
    \end{quote}
\end{frame}

\begin{frame}
    \frametitle{Defining Numbers}
    \begin{quote}
        If this possibility obtains, I shall speak, for short, of the concept $F$ being \emph{equinumerous} to the concept $G$, but I must ask that this word be regarded as an arbitrarily chosen form of expression, whose meaning is to be gleaned not fromits linguistic construction but from this stipulation.

        I therefore offer the definition

        the Number that belongs to the concept $F$ is the extension of the concept 'equinumerous to the concept $F$'. (Gl, \S68)
    \end{quote}

\end{frame}

\Section{Grundgesetze}

\begin{frame}
    \frametitle{Basic Law V}

\end{frame}

\begin{frame}
    \frametitle{Russell's Paradox}

    

\end{frame}

\End

\begin{frame}[plain, standout]
      \centering\large
      Thank you for your attention!
      
\end{frame}

\end{document}
