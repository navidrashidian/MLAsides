\documentclass[11pt]{scrartcl}


%%%%%%%%%%%%%%%%%%%%%%%%%%%%%%%%%%%%%%%%%%%%%%%%%%%%%%%%%%%%%%%%%%%%%%%%%%%%%%%%
% Make this compatible with pdflatex and lualatex, xelatex.
%%%%%%%%%%%%%%%%%%%%%%%%%%%%%%%%%%%%%%%%%%%%%%%%%%%%%%%%%%%%%%%%%%%%%%%%%%%%%%%%

\usepackage{ifxetex,ifluatex}
\newif\ifxetexorluatex
\ifxetex
  \xetexorluatextrue
\else
  \ifluatex
    \xetexorluatextrue
  \else
    \xetexorluatexfalse
  \fi
\fi

\ifxetexorluatex
  \usepackage{fontspec}
  \usepackage{xltxtra} % Not that important nowa
  % Font packages, and other xetex specific configuration here.
\else
  % Use utf8 as encoding.
  \usepackage[utf8]{inputenc}
  \usepackage[T1]{fontenc}
  % font packages
  \usepackage{lmodern} % Better than default computer modern?
\fi

%%%%%%%%%%%%%%%%%%%%%%%%%%%%%%%%%%%%%%%%%%%%%%%%%%%%%%%%%%%%%%%%%%%%%%%%%%%%%%%%
% End of pdflatex/xelatex/lualatex specific configuration
%%%%%%%%%%%%%%%%%%%%%%%%%%%%%%%%%%%%%%%%%%%%%%%%%%%%%%%%%%%%%%%%%%%%%%%%%%%%%%%%

\usepackage[hmargin=2cm,vmargin=2.5cm]{geometry} % Set small margins for handout

% BibTeX setup
\usepackage[backend=bibtex, bibstyle=alphabetic, citestyle=alphabetic]{biblatex}
\bibliography{samplebibfile}

\usepackage[english]{babel} % English
% \usepackage[ngerman]{babel} % German

\usepackage{enumitem} % Package for handling of lists
\setlist{noitemsep}   % No separating whitespace between list items.

% Standard packages for math-related things.
\usepackage{amsmath}
\usepackage{amssymb}
\usepackage{amsthm}

\usepackage{graphicx} % to include graphics with \includegraphics[options]{imagefile}

% Hyperref is great, but sometimes there are problems if packages are loaded before or after hyperref. Check the documentation.
\usepackage{hyperref} % For PDF links, toc, etc.

% Formatting of paragraphs
\parindent 0cm                     % No intendation at the beginning of a paragraph
\parskip1.5ex plus0.5ex minus0.5ex % Vertical space between two paragraphs


%%%%%%%%%%%%%%%%%%%%%%%%%%%%%%%%%%%%%%%%%%%%%%%%%%%%%%%%%%%%%%%%%%%%%%%%%%%%%%%%
% Theorem and style definitions (for amsthm). For details see
% https://en.wikibooks.org/wiki/LaTeX/Theorems
%%%%%%%%%%%%%%%%%%%%%%%%%%%%%%%%%%%%%%%%%%%%%%%%%%%%%%%%%%%%%%%%%%%%%%%%%%%%%%%%

\theoremstyle{plain} % Usual style for theorems, etc.

% All numbered with the same counter (theorem).
% \newtheorem{theorem}{Theorem}[section] % This uses numbering SECTION.COUNT instead of COUNT. Useful in longer documents.
\newtheorem{theorem}{Theorem} % Number linearly (no SECTION prefix).
% Usage: \newtheorem{environmentname}[counter]{displayedtext}
\newtheorem{proposition}[theorem]{Proposition}
\newtheorem{lemma}[theorem]{Lemma}
\newtheorem{corollary}[theorem]{Corollary}
\newtheorem*{lemma*}{Lemma} % not numbered.

\theoremstyle{definition} % Usual style definitions.
\newtheorem{definition}[theorem]{Definition}

\theoremstyle{remark} % Usual style definitions.
\newtheorem{remark}[theorem]{Remark}
\newtheorem{example}[theorem]{Example}


%%%%%%%%%%%%%%%%%%%%%%%%%%%%%%%%%%%%%%%%%%%%%%%%%%%%%%%%%%%%%%%%%%%%%%%%%%%%%%%%
% Some commands
%%%%%%%%%%%%%%%%%%%%%%%%%%%%%%%%%%%%%%%%%%%%%%%%%%%%%%%%%%%%%%%%%%%%%%%%%%%%%%%%

\newcommand{\IN}{\mathbb{N}} % blackboard bold N for natural numbers
\newcommand{\IR}{\mathbb{R}} % blackboard bold R for real numbers
\newcommand{\IZ}{\mathbb{Z}} % blackboard bold Z for integers
\newcommand{\IF}{\mathbb{F}} % blackboard bold F for fields
\newcommand{\IG}{\mathbb{G}} % blackboard bold G for groups

\newcommand{\Adversary}{\mathcal{A}} % calligraphic A for the adversary

\newcommand{\mathalgofont}[1]{\mathsf{#1}} % Define mathalgofont, as basically an alias for \mathsf. This allows to switch the font of all algorithms at once.
\newcommand{\Gen}{\mathalgofont{Gen}}
\newcommand{\Enc}{\mathalgofont{Enc}}
\newcommand{\Dec}{\mathalgofont{Dec}}

% feel free to add more commands here



%%%%%%%%%%%%%%%%%%%%%%%%%%%%%%%%%%%%%%%%%%%%%%%%%%%%%%%%%%%%%%%%%%%%%%%%%%%%%%%%
% Content begins here
%%%%%%%%%%%%%%%%%%%%%%%%%%%%%%%%%%%%%%%%%%%%%%%%%%%%%%%%%%%%%%%%%%%%%%%%%%%%%%%%

% Titlepage setup
\title{TITLE HERE} % Put yout topic name here
\subtitle{SEMINAR NAME HERE (SEMESTER HERE)} % Put name of seminar and the current semester here. For example: "Seminar on interesting questions (SS 18)"
\author{AUTHOR NAME HERE} % Put your name here
\date{DATE HERE} % The date of your presentation


% Start of document
\begin{document}
%\input{KITheader_german.tex} % german header
{ % group this
\footnotesize\sffamily
\newdimen\addresswidth
\addresswidth=\textwidth
\advance\addresswidth by -3cm
\noindent\begin{tabular}{@{}c@{}c@{}}
\parbox{\addresswidth}{Karlsruhe Institute of Technology\\Institute of Theoretical Informatics\\Research Group Cryptography and IT Security} & \parbox{3cm}{\includegraphics[width=3cm]{./logos/kit-en.pdf}}
\end{tabular}
} % end of the group

\vspace{\baselineskip}

\begin{center}
\makeatletter
{\Large\bfseries\@title\\}
\vspace{\baselineskip}
{
\@author\\
\@subtitle{}\\
\@date
\makeatother
}
\end{center}

\vspace{\baselineskip}
 % english header
%\maketitle % standard header (without KIT logo, etc)


\begin{abstract}
ABSTRACT HERE, i.e., a short summary of your topic.
\end{abstract}


\section{A section}
\label{sec:a-section}

Some text in a section.

\section*{An unnumbered section}

This section, unlike section~\ref{sec:a-section} is not numbered.
This is indicated by the ``*'', i.e., \verb|\section*| instead of \verb|\section|.
Such ``starred'' and ``unstarred'' variants also exist for many environments (and commands).
There are subsections, subsubsections and paragraphs (and subparagraphs) for more fine-grained subdivision.

\subsection*{A subsection}

Some text.

\subsubsection*{A subsubsection}

More text.

\paragraph*{A paragraph}

Even more text.


\section{Important hint}

If you don't know \LaTeX{}, how to use math mode, environment, and so on,
a short tutorial/introduction will be more useful than this document.
This document is only a reminder of some basic \LaTeX{} stuff.
Not the absolute basics, and nothing advanced.

Furthermore, use a \textbf{suitable editor}.
This will make using \LaTeX{} \emph{much} easier.

\section{Basic usage of \LaTeX}
\label{sec:math-and-env}


\subsection{{\texttt{\textbackslash{}newcommand} for shorthands}}
\label{subsec:newcommand}

Using \verb|\newcommand{\cmdname}{output}| you can define shorthands,
e.g., \verb|\IN| for natural numbers $\IN$, and so on.
See the preamble.
Much more than mere text insertion is possible.
See (advanced) \LaTeX{} introductions.

\subsection{Lists}
\label{subsec:lists}

Use \verb|itemize|, \verb|enumerate| or \verb|description| for lists.
For example \verb|itemize|:
\begin{itemize}
  \item This is the first item
  \item Now comes the second item
\end{itemize}

For example \verb|enumerate|:
\begin{enumerate}
  \item This is the first item
  \item Now comes the second item
\end{enumerate}

For example \verb|description|:
\begin{description}
  \item[First item:] This is the first item
  \item[Other stuff:] Now comes the second item
\end{description}


\subsection{Math environments}
\label{subsec:math-env}

Use \emph{inline} math mode (i.e., tex code in \$'s) for inline math,
e.g., $\lim_{x \to 1} e^{2 \pi i x} = 1$ or $\alpha \beta = \Gamma$.
Use the \verb|equation*| environment (or \verb|\[| and \verb|\]|)
for math in \emph{display} mode, e.g.,
\begin{equation*}
  \lim_{x \to 1} e^{2 \pi i x} = 1.
\end{equation*}
If you use \verb|equation| you get numbered equations which you can reference to (see section~\ref{sec:references}).
\begin{equation}
  \label{eq:eqn}
  |x| = \left\{
  \begin{array}{rl}
    -x &\mbox{ if $x<0$} \\
    x &\mbox{ otherwise}
  \end{array}
  \right.
\end{equation}

Proofs, Definition, Lemmata, Propositions, Theorems, Remarks, etc, have their own environments.
Environments can be be changed/redefined and new ones can be defined.

\begin{definition}[Negligible function]
  \label{def:negl}
  Let $f \colon \IN \to \IR$ be a function.
  If for any $k > 0$, it holds that $f(x) x^k \rightarrow 0$ for $x \to \infty$,
  then we call $f$ \emph{negligible}.
\end{definition}

\begin{remark}
  Writing $g \colon X \to Y$ looks good, while $g: X \to Y$ treats ``:'' as a \emph{division}, and looks strange.
  So use \verb|\colon| if you need a colon in math mode.
\end{remark}

\begin{lemma}
  $f(x) = 0$ is a negligible function.
\end{lemma}

\begin{proposition}
  If $f$ and $g$ are negligible functions,
  then $f + g$ is a negligible function.
\end{proposition}

The following theorem has a proof included.

\begin{theorem}[The ring of negligible functions]
  The set of negligible functions is closed under addition, subtraction and multiplication.
\end{theorem}
\begin{proof}
  This is just an application of theorems about limits of series and induction.
  For example, $0 = 0 + 0 = \lim_{x \to \infty} f(x)x^k + \lim_{x \to \infty} g(x)x^k = \lim_{x \to \infty} (f(x) + g(x))x^k$
  and
  \begin{equation*}
    0 = 0 \cdot 0 = \lim_{x \to \infty} f(x) x^{k} \lim_{x \to \infty} f(x) g(x)) x^{k} = \lim_{x \to \infty} (f(x) g(x)) x^{2k}
  \end{equation*}
\end{proof}

\begin{corollary}
  Working with negligible functions is easy.
\end{corollary}

\begin{remark}
  If $f$ is \emph{not} negligible,
  this does \emph{not} imply that $|f(x)| \geq x^{-k}$ always.
  This needs to hold for infinitely many $x \in \IN$ (and $k > 0$).
  E.g., $f(x) = 1 - (-1)^x$ is not negligible.
\end{remark}

There are a lot more useful things for math layout,
e.g., \verb|align| and \verb|aligned| environments for equations, and so on.


\subsection{Font choices (for algorithms etc.)}
\label{subsec:font-choices}

It is important to use math mode ``correctly''.
Typed letters are interpreted as individual symbols, even if they are not separated by whitespace.
For example $CTR$ looks strange,
so use $\mathit{CTR}$ or $\mathrm{CTR}$ instead,
which treats ``CTR'' as one word.
The command \verb|\mathit| uses italics, while \verb|\mathrm| does not.
There are also \verb|\mathcal| for calligraphic (e.g. $\mathcal{B}$),
and \verb|\mathsf| for sans-serif (e.g. $\mathsf{Sans}$),
and \verb|\mathbb| for blackboard bold (e.g. $\mathbb{N}$),
and \verb|\mathtt| for monospaced (e.g. $\mathtt{typewriter}$),
and so on.

It is a good idea \emph{not to} use these macros everytime,
but to define new macros instead,
which carry the \emph{semantics}.
E.g. defining a macro \verb|\IN| which prints $\IN$.
This is easier to type, read, and change.
See the preamble for some predefined examples.

Oftentimes, \verb|\mathsf| is used to typeset algorithms.
Following the advice above, the preamble defines a \verb|\mathalgofont| macro,
which is used in the definitions of \verb|\Gen|, \verb|\Enc|, \verb|\Dec|.
Thus, changing \verb|\mathalgofont|, affects all (three) macros,
ensuring consistency.

\begin{example}
  This shows the difference suitable macros make:
  \begin{enumerate}
    \item $(Gen, Enc, Dec)$ is correct if for all $k \gets Gen()$
    and all messages $m$, we have $Dec(k, Enc(k, m)) = m$.
    \item $(\Gen, \Enc, \Dec)$ is correct if for all $k \gets \Gen()$
    and all messages $m$, we have $\Dec(k, \Enc(k, m)) = m$.
  \end{enumerate}
\end{example}




\subsection{References}
\label{sec:references}

To refer to a section, or any other ``referrable'' object, use the ``ref'' command.
For example: Section~\ref{sec:a-section} or Definition~\ref{def:negl} or Eq.~(\ref{eq:eqn}).
The tilde \verb|~| is an ``unbreakable space''.
(There are more advanced ways to do this, which are especially useful for longer documents.
For example the ``cleveref'' package.)


\subsection{Literature and BibTeX}
\label{subsec:literature}

To refer to literature, use BibTeX (which needs a bib-file) and run it (on the main document).
The bibfile is set by \verb|\bibliography{FILENAME}| in the preamble.
The literature inserted by the \verb|\printbibliography{}| command. (See the end of the document).
A good source of (sample) bibfiles is \url{https://dblp.uni-trier.de}.
Use \verb|\cite{}| (e.g., \cite{DH76} or \cite{RSA78}) or variants (see Section~\ref{subsec:packages}).


\subsection{Packages and the Internet}
\label{subsec:packages}

There is a huge supply of useful packages.
For almost every problem, there's a package to solve it.
Just use the Internet to find them.
(Special mention: TikZ, cleveref, cryptocode)

Reading some short introduction/tutorial on \LaTeX{} is also recommended.
Because this document only scratches the surface:
It does not have tables, pictures, splitting the document into multiple files, and so on.
Good starting points are:
Search engines, \url{https://en.wikibooks.org/wiki/LaTeX}, \url{https://tex.stackexchange.com}



\subsection{Miscellaneous}
\label{subsec:misc}

To start a new paragraph, use either an empty line in the source tex file or the command \verb|\par|.

Footnotes work via the \verb|\footnote{}| command.\footnote{This is a footnote.}





%%%%%%%%%%%%%%%%%%%%%%%%%%%%%%%%%%%%%%%%%%%%%%%%%%%%%%%%%%%%%%%%%%%%%%%%%%%%%%%%
% Content ends here
%%%%%%%%%%%%%%%%%%%%%%%%%%%%%%%%%%%%%%%%%%%%%%%%%%%%%%%%%%%%%%%%%%%%%%%%%%%%%%%%
\printbibliography{}
\end{document}

